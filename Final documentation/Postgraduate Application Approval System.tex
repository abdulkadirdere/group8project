\documentclass{article}
\usepackage{graphicx, tabu, hyperref}
\usepackage[margin=2.5cm]{geometry}

\begin{document}
\begin{titlepage}
    \begin{center}
        \vspace*{1cm}
        \huge COMS3002 Software Engineering \\
        \LARGE Project 11 - Postgraduate Application Approval System
        
		\vspace{1.5cm}        
		
		\includegraphics[scale=0.5]{witsLogo.png} \\		
		\vspace{1.5cm}
        \textbf{Group 8} \\
        \large Abdulkadir Dere - 752817\\
        Jesse Wright - 721386 \\
        Liam Leibrandt - 814078\\
        Brenda Lin - 747243 \\
        
		\vspace{1.5cm} 
		       
                
        School of Computer Science\\
        University of Witwatersrand\\
        2 October 2017
        
    \end{center}
\end{titlepage}

\tableofcontents

\pagebreak

\section{Introduction}
\subsection{Glossary}
\begin{tabu} to \textwidth {| X[l] | X[l] |}
\hline
\textbf{Term/Acronym/Abbreviation} & \textbf{Description/Definition} \\
\hline
PGO & Postgraduate Officer \\
\hline
PGC & Postgraduate Coordinator \\
\hline
PGFO & Postgraduate Faculty Officer \\
\hline
Evaluator & Person who is responsible of evaluating the application for the recommendation phase. \\
\hline
EIE & The School of Electrical and Information Engineering \\
\hline
PAAS & Postgraduate Application Approval System \\
\hline
SIMS & Students Information Management System \\
\hline
Applicant & User who registers on the system to apply (formal request) for postgraduate degree \\
\hline
Application & Formal request submitted by the applicant to apply for a postgraduate degree \\
\hline
Associated Documentation & Any documentation that is associated with the application form. Retrieved from SIMS. These documents may also need to be analysed with the application by the users. \\
\hline
CRUD & Create, Read (View), Update (Edit) and Delete (Archive). Used to manage entities in the system \\
\hline
UX & User Experience \\
\hline
MVC & Model View Controller \\
\hline
IIS & Internet Information Service \\
\hline
\end{tabu}
\subsection{Problem Statement and Project Purpose}
The purpose of the Postgraduate Application Approval System is to provide an efficient, paperless, online application system to the school of Electical and Information Engineering at the University of Witwatersrand. The current system is a paper based system, dependent on manually passing application documents between staff and student alike. This process can be inefficient, unreliable as well as taxing on the environment in paper usage, making the application process highly unappealing and cumbersome to prospective students. The PAAS system is a much needed solution and will rectify any such dependencies that may have a negative affect on the application process.
\subsection{Project Overview}
Our aim for the project is to create an online postgraduate application approval system for the school of electrical and information engineering. This system will be completely paperless to keep the paperwork of the activity to a minimum.
The PGO will receive completed applications from students .
These applications will be checked to make sure they are ready to process. Once they are checked,
the applications with the required information can be sent to one of the three users that will either
recommend or not recommend an application. The three actors are the Research Group Lead, Identified Supervisor and the PGC. The application will be sent to either one of these actors based on the program
that the application is for. If an interview is needed, one of the three actors can book an interview
with the applicant. After the interview the user can recommend/not recommend the application. The
application will then be sent to the PGC who will then accept or reject the application based on the
the application being recommended or not and based on faculty rules and regulations. The PGFO
will receive an email/notification about the application's status. The applicant and the schools PGO
will also receive an email notifying them whether the application was accepted or rejected with an explanation.
\subsection{Summary of Benefits}
\begin{itemize}
\item Automates the application approval system.
\item Reduces delays for the application process.
\item Environmentally friendly.
\item Keeps track of the process.
\item Stores the application and other relevant data digitally.
\item Eliminates loss of documents.
\end{itemize}
\section{Software Requirement Specification}
\subsection{Overall Description}
\subsubsection{Product Perspective}
The solution we are developing will be a web application. This web application will be used by the employees of the EIE who are responsible for the postgraduate approval process of applicants to their graduate program. \\
Our solution, the Postgraduate Application Approval System (PAAS), will provide these employees with an almost completely paperless electronic way of approving postgraduate applicants. \\
The PAAS will be designed to:
\begin{itemize}
\item Send notification emails to PGO about applications that need to be processed.
\item Receive and view applications and associated documents.
\item Forward documents to Evaluator (Research Group Leads, Identified Supervisors or PGCs).
\item Create interviews for applicant and notify them by email.
\item Allow applications to be recommended by Evaluators.
\item Send application to PGC.
\item Allow PGC to accept/decline application.
\item Send the accepted/declined applications back to PGO.
\item Send notification email to PGFO.
\item Send email to applicant whether he/she has been accepted.
\item Print documents if needed at any time.
\item Login users.
\item Create users.
\item Update users if needed.
\end{itemize}
\subsubsection{Requirements Gathering}
Brainstorming: We got together as a group and identifying as many possible solutions to the problem that the EIE is facing. We then simplified the solution details. Brainstorming helps casts a broad net, determining various discrete possibilities. Then simplifying and prioritizing the details of the solution. [2] \\ \\
Observation: We were given a step-by-step walkthrough of the business process, which we believe is a more subjective form of obtaining requirements than pure observation. We then took those steps and converted them into functions for the PAAS. [2]
\subsubsection{Use Cases}
We will be converting what the PAAS is designed to do into use cases. \\
Main Use Case List: 
\begin{itemize}
\item Create Application
\item Read Document
\item Create Interview
\item Recommend Application
\item Accept Application
\item Login User \\
\end{itemize}
Secondary Use Case List: 
\begin{itemize}
\item Print Document \\
\end{itemize}
CRUD (Create, Read [View], Update [Edit], Delete [Archive]) Use Case List:
\begin{itemize}
\item ie. Manage PGO = Create PGO, View PGO, Update PGO, Archive PGO
\item Manage PGO 
\item Manage PGC
\item Manage PGFO
\item Manage Evaluator (Research Group Lead or Identified Supervisor)
\item Manage Application
\item Manage Interview
\item Manage Document
\end{itemize}
\subsubsection{User Characteristics}
The users are the people and other systems that interact with the PAAS system. A user can be primary user or a secondary user. A primary user interacts directly with the PAAS and a secondary user interacts with the PAAS indirectly. \\ \\
User List: \\
\begin{tabular} {| m{1.5cm} | m{3.5cm} | m{9.5cm} |}
\hline
\textbf{User} & \textbf{Primary/Secondary} & \textbf{Interaction with PAAS} \\
\hline
PGO & Primary & \begin{itemize} \itemsep0em
\item Receives email from PAAS about applications for processing.
\item Gets redirected to SIMS.
\item View applications and associated documents.
\item Forward documents to Evaluators.
\item Send notification email to PGFO.
\item Ability to print documents. 
\end{itemize} \\
\hline
Evaluator & Primary & \begin{itemize} \itemsep0em
\item Receive documents from PGO.
\item View applications and associated documents.
\item Setup applicant interviews.
\item Recommend/Don't recommend application.
\item Send documents to PGC and PGO.
\item Ability to print documents.
\end{itemize} \\
\hline
PGC & Primary & \begin{itemize} \itemsep0em
\item Receive documents from PGO and Evaluators.
\item View applications and associated documents.
\item Accept/Reject application.
\item Send documents to PGO.
\item Ability to print documents.
\end{itemize} \\
\hline
PGFO & Secondary & \begin{itemize} \itemsep0em
\item Receive email notifications from PGO.
\item Ability to print documents.
\end{itemize} \\
\hline
Applicant & Primary & \begin{itemize} \itemsep0em
\item Receives interview emails.
\item Receives email about application status.
\end{itemize} \\
\hline
Emailing System & Secondary & \begin{itemize}
\itemsep0em \item Sends notification emails to the relevant actors. 
\end{itemize} \\
\hline
\end{tabular}
\subsubsection{General Constraints}
\textbf{Implementation} \\ 
Not all internet browsers may work with our system. Moving from manual to digital may be time consuming, and are subject to human error. The number of active users may start out small due to human resistance towards new technology, especially those who are not computer savvy. Teaching new users how to use the system will be time-consuming. \\ \\
Due to time constraints and the fact that we are students, the system may not be fully-functional as planned. \\ \\ \\
\textbf{Hardware} \\
Any device that makes use of a supported browser will be able to use the system. We cannot guarantee that all devices will be supported. \\ \\
The system will require an internet connection. \\ \\ \\
\textbf{Software}\\
One needs a supported browser. The application will be available to access from computer and mobile devices using a browser. \\ \\ \\
\textbf{Legal Issues} \\
To obtain a web domain. The source code will belong to the University and therefore, if the client wants the rights to the source code, they might have to go through legal protocols to obtain it from Wits University. \\ \\
As students we may not be given permission to access SIMS. \\ \\ \\
\textbf{Reliability and Fault Tolerance} \\
The system needs to be reliable and should be able to recover the student documents. It is extremely frustrating for applicants to re-upload applications because of the unreliability of the system.  \\ \\
The system also needs to have as little faults as possible, since we are working with an important process at the university, this process cannot be put on hold because of a faulty system. \\ \\ \\
\textbf{Security} \\
The system is working with sensitive information and cannot be compromised. Student details and marks are very private pieces of data and cannot be leaked because of a poorly designed system. \\ \\ \\
\textbf{User} \\
Based on the security issue mentioned above, users will only be able to access the system with a username and password. Therefore users should not have access to other users' data. \\ \\
The PGO should not have access to make the final decision until the recommendation for the application is received from the relevant users. \\ 
\subsubsection{Assumptions and Dependencies}
\begin{itemize}
\item We are assuming all users have a supported browser.
\item We are assuming that all applicants are Wits students (because of time constraints we are not regarding non-Wits student applicants).
\item We are assuming all users are computer literate. 
\item The system will be dependent on a local database.
\item We are assuming that all applicants and users use email actively.
\item We are assuming that all users may need to print the application documents.
\item We are assuming that PGO is in charge of creating users.
\end{itemize}
\subsection{Detailed Requirements}
\subsubsection{External Interface Requirements} 
\textbf{Interfaces} \\
The user interfaces may be different depending on what type of user is logged into the system. But all interfaces will follow some fundamental UX principles.
Some of these UX principles are digestibility, clarity, trust, familiarity and delight. Digestibility gives the user the feeling of “I get it”. The format, components and layout of the interface should be as clear as possible so that the user can have a feeling knowing exactly what to do because of past experiences and familiarity. Clarity is used in terms of the components, fields, layout, validation, error messages and format. The formats, validation and error messages have to be clear in terms of language, ie. “the field requires a valid email address”. A user should never feel unsure when entering their details. The use of components such as date-time picker gives the user a feeling clarity and trust. The users of the PAAS should have a feeling of familiarity from the previous forms that used to fill in manually. The electronic forms should be designed around the manual forms, the formats and positions need to be as similar as possible to allow for an easier transition. A user should have a feeling of delight when using the system, they should never feel frustrated because this will lead to the users being reluctant to using the system. \cite{product-design} \\ \\
\textbf{Hardware Interfaces} \\
Since this solution is a web-based application, the hardware devices used must support the use of web browsers, as well as the ability to display a GUI and process input from the user in order to perform the interactions between client and server. To display the GUI of the application, a display device must be used, preferably with a DPI (dots per inch) above 300. If the DPI of the device is too low, the GUI may be too pixelated to view or give meaning to. For input, a keyboard is required. It may be a digitally displayed keyboard (on the display of a device) or a physical external keyboard. The keyboard is required for basic functionality of the application. Also on the aspect of input, a mouse or trackpad is required in order to perform basic mouse down functions as well as cursor movement. The device must have sufficient processing power and memory in order to run the web browser which will be the host of the web application on the device.  \\ \\
\textbf{Software Interfaces} \\
The software used for this web-based application will be web browsers. The web browsers which this application’s functionality will be tested on are FireFox, Google Chrome, Microsoft Edge and the mobile versions of these. As discussed above, the hardware devices need to be able to support FireFox, Microsoft Edge and Google Chrome web browsers.\\  \\
\textbf{Communication Interfaces} \\
The system will make use of email functionality to notify the users, both primary and secondary. The email function is used to notify applicants about the status about their application. The PGO will receive emails when there are new applications to be processed. 
\subsubsection{Functional Requirements}
\begin{tabular} {| m{5cm} | m{10cm} |}
\hline
Use Case 1: & Create Application \\
\hline
Primary Actor: & Applicant \\
\hline
Precondition: & \begin{enumerate} \itemsep0em \item The applicant must exist in the database.
\item The applicant must be logged in. 
\end{enumerate} \\
\hline
Main Success Scenario: & \begin{enumerate} \itemsep0em \item The user will request to create a new application for the system. 
\item The system will prompt the user to enter student number, first and last name, ID number, email, contact number, school, faculty, street number, street name, suburb, city, province.
\item The system will prompt the user to confirm the selection.
\item The user will confirm.
\item The system will prompt the user to upload their  documentation.
\item The system will prompt the user to confirm the selection.
\item The user will confirm.
\item The system will notify the user with a success message that the user has successfully created an application.
\end{enumerate}\\
\hline
Exception Scenarios: & \begin{enumerate}
\itemsep0em \item The student number exists, if so the user will be displayed a message indicating that they have already registered.
\item An error message will be displayed and system will redirect the user to the home page.
\end{enumerate} \\
\hline
\end{tabular}
\\ \\ \\ \\
\begin{tabular} {| m{5cm} | m{10cm} |}
\hline
Use Case 2: & Read Document \\
\hline
Primary Actor: & PGO, Evaluator, PGC \\
\hline
Precondition: & \begin{enumerate} \itemsep0em \item The user must be logged in. \item The user must have an application that needs to be processed and/or the associated documents. \end{enumerate} \\
\hline
Main Success Scenario: & \begin{enumerate} \itemsep0em \item The user will prompt the system to view a certain document. \item The system will open the document to be viewed. \end{enumerate} \\
\hline
Exception Scenarios: & None. \\
\hline
\end{tabular}
\\ \\ \\ \\
\begin{tabular} {| m{5cm} | m{10cm} |}
\hline
Use Case 3: & Create Interview \\
\hline
Primary Actor: & Evaluator (any relevant actors - PGC, research group lead or identified supervisor) \\
\hline
Secondary Actor: & Applicant \\
\hline
Precondition: & \begin{enumerate} \itemsep0em \item The user must be logged in.
\item The user must have an application that needs to be processed.
\item The application should not be evaluated.
\end{enumerate} \\
\hline
Main Success Scenario: & \begin{enumerate} \itemsep0em \item The user will prompt the system that they want to setup an interview with the applicant.
\item The system will open the interview form.
\item The system will prompt the user to enter in the details of the interview such as a date, time and venue.
\item The user will fill in these details.
\item The system will prompt the user to confirm the interview details.
\item The user will confirm the details.
\item The system will notify the user that the interview creation was successful.
\item The system will redirect the user to the home page.
\end{enumerate} \\
\hline
Exception Scenarios: & \begin{enumerate} \itemsep0em \item The user will not confirm the details of the interview.
\item The system will keep the details intact, since the user could have made a small mistake that the user needs to change.
\end{enumerate} \begin{enumerate} \itemsep0em \item  The system will notify the user that the interview creation was not successful.
\item The system will redirect the user back to the interview form.
\end{enumerate}\\
\hline
\end{tabular}
\\ \\ \\ \\
\begin{tabular} {| m{5cm} | m{10cm} |}
\hline
Use Case 4: & Recommend Application \\
\hline
Primary Actor: & Evaluator, PGC \\
\hline
Precondition: & \begin{enumerate} \itemsep0em \item The user must be logged in. 
\item The user must have an application that needs to be processed.
\item The application should not be evaluated.
\end{enumerate} \\
\hline
Main Success Scenario: & \begin{enumerate} \itemsep0em \item The user will prompt the system to a recommend an application.
\item The user will enter the recommendation description.
\item The system will prompt the user to confirm the recommendation.
\item The user will confirm the recommendation.
\item The system will send the application to PGC for final decision.
\end{enumerate}\\
\hline
Exception Scenarios: & \begin{enumerate}  \itemsep0em \item The user will not confirm the recommendation.
\item The system will redirect back to the previous screen.
\end{enumerate} \\
\hline
\end{tabular}
\\ \\ \\ \\
\begin{tabular} {| m{5cm} | m{10cm} |}
\hline
Use Case 5: & Finalize Application \\
\hline
Primary Actor: & PGC \\
\hline
Precondition: & \begin{enumerate} \itemsep0em \item The user must be logged in. 
\item The user must have an application that needs to be processed.
\item The application should be evaluated.
\item The application should not be accepted/rejected.
\end{enumerate} \\
\hline
Main Success Scenario: & \begin{enumerate} \itemsep0em \item The user will prompt the system to accept/reject an application.
\item The system will prompt the user to confirm the acception/rejection.
\item The user will confirm the acception/rejection.
\end{enumerate}\\
\hline
Exception Scenarios: & None.\\
\hline
\end{tabular}
\\ \\ \\ \\
\begin{tabular} {| m{5cm} | m{10cm} |}
\hline
Use Case 6: & Print Document \\
\hline
Primary Actor: & PGO, Evaluator, PGC, PGFO \\
\hline
Precondition: & \begin{enumerate} \itemsep0em \item The user must be logged in. 
\item The user must have an application that needs to be processed and/or the associated documents.
\end{enumerate} \\
\hline
Main Success Scenario: & \begin{enumerate} \itemsep0em \item The user will prompt the system to print a certain document.
\item The system will print the document.
\end{enumerate}\\
\hline
Exception Scenarios: & None. \\
\hline
\end{tabular}
\\ \\ \\ \\
\begin{tabular} {| m{5cm} | m{10cm} |}
\hline
Use Case 7: & Login User \\
\hline
Primary Actor: & PGO, Evaluator, PGC, PGFO \\
\hline
Precondition: & The user cannot be logged in.  \\
\hline
Main Success Scenario: & \begin{enumerate} \itemsep0em \item The user will prompt the system to log in.
\item The system will prompt the user to enter in the email address and password.
\item The user will enter these details and login.
\item The system will direct the user to the home page.
\end{enumerate}\\
\hline
Exception Scenarios: & \begin{enumerate} \itemsep0em \item The user will enter in the incorrect details.
\item The system will show an error message and prompt the user to enter the details correctly.
\item The user will enter these details and login.
\item The system will direct the user to the home page.
\end{enumerate}\\
\hline
\end{tabular}
\\ \\ \\ \\
\begin{tabular} {| m{5cm} | m{10cm} |}
\hline
Use Case 8: & Create User \\
\hline
Primary Actor: & PGO \\
\hline
Precondition: & The user must be logged in. \\
\hline
Main Success Scenario: & \begin{enumerate} \itemsep0em \item The user will request to create a new user for the system. 
\item The system will prompt the user to insert first name, last name, email, ID number, contact number and password for the new user.
\item The system will prompt the user to confirm the selection.
\item The user will confirm.
\item The system will notify the user with a success message that the user has been successfully created.
\end{enumerate}\\
\hline
Exception Scenarios: & \begin{enumerate} \itemsep0em \item The  the first name, last name, email, ID number, contact number or password is invalid (blank).
\item An error message will be displayed and system will redirect the user to the previous page.
\end{enumerate} \begin{enumerate} \itemsep0em \item The email has already been used.
\item An error message will be displayed and system will redirect the user to the previous page.
\end{enumerate} \begin{enumerate}  \itemsep0em \item The user will not confirm the selection.
\item The system will redirect the user to the previous page. \end{enumerate} \\
\hline
\end{tabular}
\\ \\ \\ \\
\begin{tabular}{| m{5cm} | m{10cm} |}
\hline
Use Case 9: & View User \\
\hline
Primary Actor: & PGO \\
\hline
Precondition: & The user must be logged in. \\
\hline
Main Success Scenario: & \begin{enumerate} \itemsep0em \item The user will request to view a user in the system. 
\item The system will prompt the user to select a user.
\item The system will display the user details.
\end{enumerate} \\
\hline
Exception Scenarios: & None. \\
\hline
\end{tabular}
\\ \\ \\ \\
\begin{tabular}{| m{5cm} | m{10cm} |}
\hline
Use Case 10: & Update User \\
\hline
Primary Actor: & PGO, Evaluator, PGC, PGFO \\
\hline
Precondition: & The user must be logged in. \\
\hline
Main Success Scenario: & \begin{enumerate} \itemsep0em \item The user will request to update their profile. 
\item The system will display the user in the system and allow the user to edit the user’s attributes.
\item The user will enter in the details that they want to change.
\item The user will confirm these changes.
\item The system will notify the user with a success message that the user has been successfully updated.
\end{enumerate} \\
\hline
Exception Scenarios: & None. \\
\hline
\end{tabular}
\\ \\ \\ \\
\begin{tabular}{| m{5cm} | m{10cm} |}
\hline
Use Case 11: & Archive User \\
\hline
Primary Actor: & PGO \\
\hline
Precondition: & The user must be logged in. \\
\hline
Main Success Scenario: & \begin{enumerate} \itemsep0em \item The user will request to archive (delete) a user in the system. 
\item The system will prompt the user to select a user.
\item The system will notify the user with a success message that the user has been successfully archived (Deleted).
\end{enumerate} \\
\hline
Exception Scenarios: & None. \\
\hline
\end{tabular}
\subsubsection{Performance Requirements}
This application is more dependent on the accuracy of communication and information than it is dependent on the overall performance and speed of the application. We aim to have a reliable platform on which communication is priority. Although performance is not negligible, it is not a requirement at the expense of a loss of accuracy in the communication between applicant and PGO. The application will not be demanding on the hardware but it will be demanding on the bandwidth available to the device. The slower the connection to the server where the database is stored, the slower the overall interaction with the web app will be.
\subsubsection{Design Constraints}
The application will not be optimized for systems that run Safari browsers since there are no such devices available to our team for testing. Running the web application on a safari browser is not recommended and will be done at the user's own risk.
\subsubsection{Software System Attributes}
Availability: The system will be running constantly.  \\ \\
Security: The system's security and reliability is mentioned in Section 2.1.5 \\ \\ 
Maintainability: PGO will be admin and be able to create and archive users. \\
After testing and feedback, the developers would be able to update the system to suit the user's needs.
\section{Design}
\subsection{Choice of a Software Development Life-Cycle}
\subsubsection{SCRUM}
SCRUM is our choice of a software development life-cycle for our project. It is an agile development method which is iterative and incremental.
This is how we plan to implement it:
\begin{itemize}
\item We will create a wish list of use cases and add them to our backlog.
\item During sprint planning, we will pull some of the use cases from the backlog and add them to our sprint backlog, and then decide how to implement those use cases.
Our sprint time is 6 - 8 weeks, depending on the team's availability. 
\item Along the way, the ScrumMaster (Project Leader) keeps the team focused on its goal.
\item At the end of the sprint, the use cases should be implemented and work to the best of its ability. 
\item The sprint ends with a sprint review and retrospective.
As the next sprint begins, we will choose more use cases from the backlog and begin working again. \cite{Scrum}
\end{itemize}
\subsection{Choice of Architecture}
\subsubsection{Three Tier Architecture}
A three-tier architecture is a programming model that enables the distribution of application functionality across three independent systems. \cite{descTier} \\
\includegraphics[scale=1]{Capture2.png} 
\begin{itemize}
\item A Presentation Layer that sends content to browsers in the form of HTML/JS/CSS. 
\item An Application Layer that uses an application server and processes the business logic for the application. This might be written in C\# or JavaScript.
\item A Data Layer which is a database management system that provides access to application data. This will be Microsoft SQL Server (IIS Server). \\
\end{itemize}
\subsubsection{Benefits}
\begin{itemize}
\item It gives you the ability to update the technology stack of one tier, without impacting other areas of the application.
It allows for team members to each work on their own areas of expertise.
\item You are able to scale the application up and out. A separate back-end tier, for example, allows you to deploy to a variety of databases instead of being locked into one particular technology. It also allows you to scale up by adding multiple web servers.
\item It adds reliability and more independence of the underlying servers or services.
\item It provides an ease of maintenance of the code base, managing presentation code and business logic separately, so that a change to business logic, for example, does not impact the presentation layer. \cite{tier}
\end{itemize}
\subsection{Front-end Interface Method}
A web application which allows for browser support will be created. The application should then work on  browsers including mobile browsers.
\subsection{Back-End Service}
ASP.net MVC uses SQL Server and we will use a local database (IIS Server) to store out details.
\subsection{Software Requirements for the Web Application}
The following software tools will be used:
\begin{itemize}
\item Visual Studio 2017
\item SQL Server 2014
\item .Net Framework 4.6.1
\end{itemize}
Please note: 
\begin{itemize}
\item These are the applications used in the production of the Web Application. The Web Application has been tested on these software with the specified versions. There may be configuration errors if the correct versions are not used.
\item MVC package must be included in Visual Studio 2017.
\item We strongly advise to run the application using Google Chrome.
\end{itemize}
\subsection{Other Supporting Software}
Bootstrap will also be used to make sure that the Web Application has a consistent theme and is compatible on all browsers. \\
\subsection{Student Responsibilities}
\begin{itemize}
\item Abdulkadir Dere - Group Leader
\item Brenda Lin - Quality Assurance
\item Jesse Wright - Technical Lead
\item Liam Leibrandt - Analysis Lead
\end{itemize}
\subsection{Sprint Plan}
\includegraphics[scale=0.5]{Sprint1.png}\\
\includegraphics[scale=0.65]{Sprint1Backlog.PNG}\\
\includegraphics[scale=0.5]{Sprint2.png}\\
\includegraphics[scale=0.65]{Sprint2Backlog.PNG}\\
\includegraphics[scale=0.5]{Sprint3a.png}\\
\includegraphics[scale=0.5]{Sprint3b.png}\\
\includegraphics[scale=0.65]{Sprint3Backlog.PNG}\\ 
\includegraphics[scale=0.5]{Sprint4a.png}\\
\includegraphics[scale=0.5]{Sprint4b.png}\\
\includegraphics[scale=0.5]{Sprint4c.png}\\
\includegraphics[scale=0.65]{Sprint4Backlog.PNG}\\
\subsection{Use Case Diagram}
\includegraphics[scale=0.65]{UseCaseDiagram.png} \\
Note: Maintenance use case in the diagram refers to all the maintenance use cases specified in section 2.2.2
\subsection{Class Model Diagram}
\includegraphics[scale=0.5]{ERD.png} \\
\subsection{Process Model (Flow Model)}
\includegraphics[scale=0.6]{ApplicationSystemFlow.png} \\ 
\subsection{Sequence Diagrams}
\includegraphics[scale=0.7]{CreateApplicationSD.png} \\ \\
\includegraphics[scale=0.7]{CreateInterviewSD.png} \\ \\
\includegraphics[scale=0.7]{FinalizeApplicationSD.png} \\ \\
\includegraphics[scale=0.7]{ReadDocumentSD.png} 
\subsection{State Machine Diagrams}
\includegraphics[scale=0.45]{ApplicationSMD.png} \\ \\
\includegraphics[scale=0.7]{InterviewSMD.png} \\ \\
\includegraphics[scale=0.75]{EmployeeSMD.png}
\section{Implementation (User Manual)}
\subsection{Create Application}
\begin{center}
\includegraphics[scale=0.5]{CreateApplication.png}\\
\includegraphics[scale=0.5]{CreateApplication2.png}\\
This is the application creation page.\\ \bigskip
\includegraphics[scale=0.5]{CreateApplication3.png}\\
\includegraphics[scale=0.5]{CreateApplication4.png}\\
Fill in the required details and click create.\\ \bigskip
\includegraphics[scale=0.5]{CreateApplication5.png}\\
Confirm your details. \\ \bigskip
\includegraphics[scale=0.5]{CreateApplication6.png}\\
\includegraphics[scale=0.5]{CreateApplication7.png}\\
Browse for application documents to submit. \\ \bigskip
\includegraphics[scale=0.5]{CreateApplication8.png}\\
\includegraphics[scale=0.5]{CreateApplication9.png}\\
Upload the document and confirm the submission. \\ \bigskip
\end{center}
\subsection{Accept Application}
\begin{center}
\includegraphics[scale=0.45]{AcceptApplication.png}\\
This is the application index list. Click on process. \\ \bigskip
\includegraphics[scale=0.5]{AcceptApplication2.png}\\
Click view application to download respective application documents.\\ \bigskip
\includegraphics[scale=0.5]{AcceptApplication3.png}\\
Save the document. \\ \bigskip
\includegraphics[scale=0.5]{AcceptApplication4.png}\\
Click accept to accept the application and confirm. \\ \bigskip
\includegraphics[scale=0.5]{AcceptApplication5.png}\\
Leave feedback on the accepted application.\\ \bigskip
\includegraphics[scale=0.5]{AcceptApplicationReject.png}\\
Click reject to reject the application and confirm.\\ \bigskip
\end{center}
\subsection{Recommend Application}
\begin{center}
\includegraphics[scale=0.45]{RecommendApplication.png}\\
This is the application index list. Click on process.\\ \bigskip
\includegraphics[scale=0.5]{RecommendApplication2.png}\\
Click view application to download respective application documents.\\ \bigskip
\includegraphics[scale=0.5]{RecommendApplication3.png}\\
Save the document.\\ \bigskip
\includegraphics[scale=0.5]{RecommendApplication4.png}\\
Click recommend to recommend the application and confirm.\\ \bigskip
\includegraphics[scale=0.5]{RecommendApplication5.png}\\
\includegraphics[scale=0.5]{RecommendApplication6.png}\\
Leave feedback on the recommendation and confirm. \\ \bigskip
\includegraphics[scale=0.5]{RecommendApplicationNotRecommended.png}\\
Click do not recommend to not recommend the application and confirm.\\ \bigskip
\end{center}
\subsection{Create Interview}
\begin{center}
\includegraphics[scale=0.45]{CreateInterview.png}\\
This is the application index list. Click on create interview.\\ \bigskip
\includegraphics[scale=0.5]{CreateInterview2.png}\\
\includegraphics[scale=0.5]{CreateInterview3.png}\\
Fill in the appropriate details and confirm the interview slot.\\ \bigskip
\end{center}
\subsection{Process Interview}
\begin{center}
\includegraphics[scale=0.5]{ProcessInterview.png}\\
This is the interview index list. Click on complete.\\ \bigskip
\includegraphics[scale=0.5]{ProcessInterviewComplete.png}\\
Confirm that the interview has been completed.\\ \bigskip
\includegraphics[scale=0.5]{ProcessInterviewCancelled.png}\\
Click on cancel and confirm that the interview has been cancelled.\\ \bigskip
\end{center}
\subsection{Verify Application}
\begin{center}
\includegraphics[scale=0.5]{VerifyApplication.png}\\
This is the application verification index list. Click verify.\\ \bigskip
\includegraphics[scale=0.5]{VerifyApplication2.png}\\
Click view application to download respective application documents.\\ \bigskip
\includegraphics[scale=0.5]{VerifyApplication3.png}\\
Save the document.\\ \bigskip
\includegraphics[scale=0.5]{VerifyApplication4.png}\\
Click verify and confirm the verification.\\ \bigskip
\end{center}
\section{Testing}
\subsection{Functionality Testing}
\subsubsection{Motivation for Functionality Testing} 
Functionality testing is very important because it allows us to check and ensure that all functions that we have implemented for our system’s use
cases, are running correctly. This ensures that the data passed from users through the functions (via field forms) to the database is accurate and
without errors. This also allows us to determine if any use cases/processes contain faulty logic or flow so that we may review and alter them. 
\subsubsection{Case Name: Create Application} 
Requirement Description: The student should be able to create an application. \\
\begin{tabu} to \textwidth {| X[l] | X[l] | X[l] | X[l] | X[l] | X[l] | X[l]|}
\hline
\textbf{Test Number} & \textbf{Action/ Task} & \textbf{Test Input} & \textbf{Expected Results} & \textbf{Actual Results} & \textbf{Pass/Fail} & \textbf{Comments} \\
\hline
1 & Applicant will prompt the system to create a new application. & & Application form opens. & The system did as expected. & Pass & \\
\hline
2 & The system will prompt the user to enter student number,
first and last name, ID number, email, contact number,
school, faculty, street number, street name, suburb, city and
province. & student number, first and last name, ID number, email, contact number, school, faculty, street number, street name, suburb, city and
province & The form will be filled with applicant’s details. & The system did as expected. & Pass & \\
\hline
3 & The system will prompt the user to upload their documentation in a .pdf format. & Documenta- tion in pdf format. & The document(s) will be uploaded. & The system did as expected. & Pass & \\
\hline
4 & The user will confirm the selection after the system has prompted the user. & & The creation of the application will be confirmed. & The system did as expected. & Pass & \\
\hline
\end{tabu}
\subsubsection{Case Name: Read Document} 
Requirement Description: The user should be able to download the a required document. \\
\begin{tabu} to \textwidth {| X[l] | X[l] | X[l] | X[l] | X[l] | X[l] | X[l]|}
\hline
\textbf{Test Number} & \textbf{Action/ Task} & \textbf{Test Input} & \textbf{Expected Results} & \textbf{Actual Results} & \textbf{Pass/Fail} & \textbf{Comments} \\
\hline
1 & The actor will select the document to be viewed. & & The document will be downloaded. & The system did as expected. & Pass & \\
\hline
\end{tabu}
\subsubsection{Case Name: Create Interview} 
Requirement Description: The user should be able to request an interview with the applicant. \\
\begin{tabu} to \textwidth {| X[l] | X[l] | X[l] | X[l] | X[l] | X[l] | X[l]|}
\hline
\textbf{Test Number} & \textbf{Action/ Task} & \textbf{Test Input} & \textbf{Expected Results} & \textbf{Actual Results} & \textbf{Pass/Fail} & \textbf{Comments} \\
\hline
1 & The actor will prompt the system for an interview form. & & The system will open the interview form. & The system did as expected. & Pass & \\
\hline
2 & The system will prompt the actor to enter date time and venue of the interview. & Date, time and venue & The form will be filled with interview’s details. & The system did as expected. & Pass & \\
\hline
3 & The user will confirm the selection after the system has prompted the user. & & The creation of the interview will be confirmed. & The system did as expected. & Pass & \\
\hline
\end{tabu}
\subsubsection{Case Name: Recommend Application} 
Requirement Description: The user should be able to recommend an application. \\
\begin{tabu} to \textwidth {| X[l] | X[l] | X[l] | X[l] | X[l] | X[l] | X[l]|}
\hline
\textbf{Test Number} & \textbf{Action/ Task} & \textbf{Test Input} & \textbf{Expected Results} & \textbf{Actual Results} & \textbf{Pass/Fail} & \textbf{Comments} \\
\hline
1 & The actor will prompt the system to recommend an application. & & The system will open the recommendation form. & The system did as expected. & Pass & \\
\hline
2 & The system will prompt the user to enter a recommendation description. & Recommen- dation description. & The form will be filled with the recommendation description. & The system did as expected. & Pass & \\
\hline
3 & The user will confirm the selection after the system has prompted the user. & & The recommendation will be confirmed. & The system did as expected. & Pass & \\
\hline
\end{tabu}
\subsubsection{Case Name: Finalize Application} 
Requirement Description: The user should be able to reject or accept an application. \\
\begin{tabu} to \textwidth {| X[l] | X[l] | X[l] | X[l] | X[l] | X[l] | X[l]|}
\hline
\textbf{Test Number} & \textbf{Action/ Task} & \textbf{Test Input} & \textbf{Expected Results} & \textbf{Actual Results} & \textbf{Pass/Fail} & \textbf{Comments} \\
\hline
1 & The actor will prompt the system to finalize an application. & & The system will open the finalization form. & The system did as expected. & Pass & \\
\hline
2 & The system will prompt the user to enter a finalization description. & Finalization description. & The system will open the finalization form. & The system did as expected. & Pass & \\
\hline
3 & The user will confirm the selection after the system has prompted the user. & & The finalization of the application will be confirmed. & The system did as expected. & Pass & \\
\hline
\end{tabu}
\subsubsection{Case Name: Login User} 
Requirement Description: The user should be able to download the a required document. \\
\begin{tabu} to \textwidth {| X[l] | X[l] | X[l] | X[l] | X[l] | X[l] | X[l]|}
\hline
\textbf{Test Number} & \textbf{Action/ Task} & \textbf{Test Input} & \textbf{Expected Results} & \textbf{Actual Results} & \textbf{Pass/Fail} & \textbf{Comments} \\
\hline
1 & The actor will prompt the system to login. & & The system will open the login page. & The system did as expected. & Pass & \\
\hline
2 & The system will prompt the user to enter in their email and password. & Email and password & The login details will be filled with the actors email and password. & The system did as expected. & Pass & \\
\hline
3 & The user will confirm the selection after the system has prompted the user. & & The user will be logged in. & The system did as expected. & Pass & \\
\hline
\end{tabu}
\subsection{Security Testing}
\subsubsection{Motivation for Security Testing} 
Security testing is the process of testing the system to highlight the flaws and bugs within the system with regards to confidentiality, integrity, authentication, authorisation, availability and non-repudiation. \\
Security testing is important as it maintains the systems intended functionality. It also analysis the system for any weaknesses, technical flaws or vulnerabilities.\subsubsection{Case Name: Security Testing for PAAS} 
Requirement Description: The system should be able to detect any flaws in confidentiality integrity, authentication, authorisation and non-repudiation. \\
\begin{tabu} to \textwidth {| X[l] | X[l] | X[l] | X[l] | X[l] | X[l] | X[l]|}
\hline
\textbf{Test Number} & \textbf{Action/ Task} & \textbf{Test Input} & \textbf{Expected Results} & \textbf{Actual Results} & \textbf{Pass/Fail} & \textbf{Comments} \\
\hline
1 & (Confiden- tiality) Enter the URL manually to access to the Create User page & \url{http://localhost:53861/PGOs/create} & User shouldn’t be able to view the content of the Create User (hence don’t have access to customer details) & Create User page is displayed. & Fail & Create User page is
displayed by bypassing the login  in screen so page is visible to the user without login in. \\
\hline
2 & (Integrity) Enter the URL manually to access to the user edit page & \url{http://localhost:53861/PGOs/Edit/1} & User shouldn’t be able to edit the content of a user without logging in & User Edit page is displayed & Fail & User Edit page has been shown to the user, so the user can edit the user details without login in \\
\hline
3 & (Authenti- cation) Enter the correct login details on the Login page and click on Login button & Email Address: \url{admin@admin.co.za} Password: admin & Login details are accepted and user can view the Staff Portal’s page & The system does as expected. & Pass & \\
\hline
4 & (Authenti- cation) Enter the wrong
login details on the Login page and click on Login button  & Email Address: \url{trkye@yahoo.com} Password: pass123 & Login details are rejected and user can’t view the Staff Portal’s page &  The system does as expected. & Pass &\\ 
\hline
5 & (Authorisation) Enter the URL manually to access to the Application Interviews & \url{http://localhost:53861/Interviews/ApplicationsInterviews} & User shouldn’t be able to see the content of the Application Interviews list without logging in &  Application Interviews page is not displayed to the user. User is asked to login. & Pass &\\ 
\hline
6 & (Availability) View the user details in
the PGO view and compare it to the PGO details in the database & \url{http://localhost:53861/PGOs/index} & PGO details in the PGO view page and
PGO details in the PGO Table should match & The system does as expected. & Pass & \\
\hline
\end{tabu}
\pagebreak
\linebreak
\begin{tabu} to \textwidth {| X[l] | X[l] | X[l] | X[l] | X[l] | X[l] | X[l]|}
\hline
7 & (Non-repudiation) Download the uploaded PDF document in the database. & & User should be able to
view if the uploaded PDF document has been created in the database & The system does as expected. & Pass & \\
\hline
\end{tabu}
\subsection{User Acceptance Testing (UAT)}
\subsubsection{Motivation for User Acceptance Testing} 
The user of the software product performs User Acceptance Testing (UAT). The acceptance criteria is specified for a given scenario to test that certain scenario. This test is important so we can evaluate the software under actual business or real-world scenarios to check if the software meets the requirements requested by the client. If a certain scenario fails the test then the results will be analysed to classify the priority of each scenario.\\ \\
This testing method will highlight if any feature of the system is not working. The team will use the results to reconfigure these processes.
\subsubsection{Case Name: Create Application} 
Requirement Description: The system should be able to validate the user input and evaluate the given use case. \\
\begin{tabu} to \textwidth {| X[l] | X[l] | X[l] | X[l] | X[l] | X[l] | X[l]|}
\hline
\textbf{Test Number} & \textbf{Action/ Task} & \textbf{Test Input} & \textbf{Expected Results} & \textbf{Actual Results} & \textbf{Pass/Fail} & \textbf{Comments} \\
\hline
1 & Fill-in all the input
boxes with valid information in Create Application & Student
Number:789654
First Name: James
Last Name: Doe
ID Number:
9403056132181
Email address:
jamesdoe@gmail.com
Contact Number:
0725896321
School: Information
Systems
Faculty: Science
Street Number: 21
Street Name: Inner
Street
Suburb: Observatory
City: Johannesburg
Province: Gauteng & The user should be
directed to the confirmation page of Create Application. & The system did as expected. & Pass & \\
\hline
2 & Do not fill-in any of the input boxes
on the Create Application page & 
Student Number:
First Name:
Last Name:
ID Number:
Email address:
Contact Number:
School:
Faculty:
Street Number:
Street Name:
Suburb:
City:
Province:
& The user will be asked
to enter the required
details with a warning
under the empty fields. & The system did as expected. & Pass & \\
\hline
\end{tabu}
\subsubsection{Case Name: Create Interview} 
Requirement Description: The system should be able to validate the user input and evaluate the given use case. \\
\begin{tabu} to \textwidth {| X[l] | X[l] | X[l] | X[l] | X[l] | X[l] | X[l]|}
\hline
\textbf{Test Number} & \textbf{Action/ Task} & \textbf{Test Input} & \textbf{Expected Results} & \textbf{Actual Results} & \textbf{Pass/Fail} & \textbf{Comments} \\
\hline
1 & Fill-in all the input
boxes with valid information in Create Interview & Date: 2017/09/30
Interview Time: 01:30PM
Venue: CLM & The user should be
directed to the confirmation page of Create Interview. & The system did as expected. & Pass & \\
\hline
2 & Do not fill-in any of the input boxes on the Create Interview Page & Date: 
Interview Time:
Venue:  & The user will be asked to enter the required details with a error warning under the empty fields. & The system did as expected. & Pass & \\
\hline
2 & Do not fill-in any of the input boxes
on the Create Application page & 
Student Number:
First Name:
Last Name:
ID Number:
Email address:
Contact Number:
School:
Faculty:
Street Number:
Street Name:
Suburb:
City:
Province:
& The user will be asked
to enter the required
details with a warning
under the empty fields. & The system did as expected. & Pass & \\
\hline
\end{tabu}
\subsection{Cross-Browser Compatibility Testing}
\subsubsection{Motivation for Cross-Browser Compatibility Testing} 
Cross browser testing is the process of testing a web application across different browsers to ensure that a web application works as intended across multiple browsers since certain components might work differently on different web browsers. \\ \\
We shall test our web application on the following web browsers: 
\begin{itemize}
\item Microsoft Edge version 40.15063.0.0
\item Google Chrome version 60.0.3112.113
\item Mozilla Firefox version 55.0.3 
\end{itemize}
Desktop Browser tests have been conducted on Windows 10 operating system. \\ \\
Mobile Browser tests have been conducted on Android 7.0 Nougat operating system. 
\subsubsection{Case Name: Check Cross-Browser Compatibility}
Requirement Description: The system should be compatible with different types of browsers. \\ \\
\begin{tabu} to \textwidth {| X[l] | X[l] | X[l] | X[l] | X[l] | X[l] | X[l]|}
\hline
\textbf{Test Number} & \textbf{Action/ Task} & \textbf{Test Input} & \textbf{Expected Results} & \textbf{Actual Results} & \textbf{Pass/Fail} & \textbf{Comments} \\
\hline
1 & Access the Web Application using the Google Chrome Desktop browser (version 60.0.3112. 113) & Run the PAAS Web Application on Google Chrome Desktop Browser (version 60.0.3112. 113) &  The user should be able to view and interact with the PAAS website, including login into the system & The system did as expected & Pass & \\
\hline
2 & Access the Web Application using the Google Chrome Mobile browser (version 60.0.3112. 116) & Run the PAAS Web Application on Google Chrome Mobile Browser (version 60.0.3112. 116) &  The user should be able to view and interact with the PAAS website, including login into the system & The system did as expected & Pass & \\
\hline
3 & Access the Web Application using the Mozilla Firefox Desktop browser (version 55.0.3) & Run the PAAS Web Application on Mozilla Firefox Desktop Browser (version 55.0.3) & The user should be able to view and interact with the PAAS website, including login into the system & The system did as expected & Pass & \\
\hline
4 & Access the Web Application using the Mozilla Firefox Mobile browser (version 55.0.2) & Run the PAAS Web Application on Mozilla Firefox Mobile Browser (version 55.0.2) & The user should be able to view and interact with the PAAS website, including login into the system & The system did as expected & Pass & \\
\hline
5 & Access the Web Application using the Microsoft Edge browser (version 40.15063.0.0) & Run the PAAS Web Application on Microsoft Edge browser (version 40.15063.0.0) & The user should be able to view and interact with PAAS website, including login into the system & Failed to display date/time picker & Fail & \\
\hline
\end{tabu}
\subsection{Test Summary}
Functionality Testis was successful for all specified use cases. The system was able to achieve all the requests from the users side. The system was able to create an application, download a document, create an interview, recommend an application and login users.\\ \\
Security Test was successful in authenticating the user when the user enters the correct and incorrect details. System can detect if the user should be allowed to access or not. Users may not manually access via the URL. Logged in users may be able to download the files requested. \\ \\
User Acceptance Testing has been conducted on specified use cases. The validity of the input details are checked. The user may not leave any boxes unfilled. \\ \\
Cross Browser Compatibility Test was successful for all the specified browsers except Microsoft Edge. Microsoft Edge does not display the Date Picker. So, the user cannot view Date Picker hence they can’t select a date. This feature works well with other browsers. After analyses and research, we have found that Microsoft Edge has default style and selector for date. The default style takes precedence over additional styling done through the plugin. The user can still enter a date manually with the format of “yyyy/mm/dd”. This is problematic as the does not know the format hence will not be able to enter the correct date format. This issue is noted and will be fixed in construction phase.

\begin{thebibliography}{9}
\bibitem{product-design}
Clark Wimberly, sitepoint, \\
\texttt{https://www.sitepoint.com/5-simple-ux-principles-guide-product-design/}

\bibitem{Scrum}
Scrum Alliance, \\
\texttt{https://www.scrumalliance.org/why-scrum}

\bibitem{tier}
Bob Pepalis, IZENDA, \\
\texttt{https://www.izenda.com/blog/5-benefits-3-tier-architecture/}

\bibitem{descTier}
IBM Knowledge Centre, \\
\texttt{https://www.ibm.com/support/knowledgecenter/en/SSAW57\_8.5.5/com.ibm.websphere.nd.doc/ \linebreak ae/covr\_3-tier.html}

\end{thebibliography}


\end{document}